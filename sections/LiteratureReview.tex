Several studies both on user profiling and on real-time automatic trip
itinerary generation have been carried out throughout the years. 

Sylejmani et al.\cite{Sylejmani2017} have defined the Tourist Trip
Design Problem(TTDP) as part of the Orienteering Problem(OP). OP
problems contain a number of nodes each containing a score and try to
solve the path containing the maximal score constrained with
parameters such as time and budget \cite{Gunawan2016}. There are many
solutions to this problem which will be discussed in the next section.
% Maybe talk about OP..etc

\subsection{Automated Trip Systems}


In 2004, a paper by Dunstall et al. \cite{DUNSTALL2008a} was published
using a prototype called the The Electronic Travel Planner (ETP). This
system selects destinations by determining activities based on the
user’s preferences. Each activity is stored in a relational database
with information such as duration, availability, date and time
categorised as either tours, lodging or transportation. The
requirements for forming such an itinerary include the number of
children and adults, the location, the date range, budget and user
preferences in the form of \emph{mandatory, at least once, desired,
forbidden and permitted} activities. Since examples given in the paper took 15-45 seconds to process
the results running time was listed as an issue.

The Recommender System (RS) was provided by Sebatsia et al.
\cite{Sebastia2009a} and Garcia et al. \cite{Garcia2011} to suggest tourist locations.
User preferences are collected in the form of age, gender, nationality
and ontology. The recommender is based on 4 techniques, \emph{Demographic
recommendation, collaborative recommendation, content based
recommendation and knowledge based recommendation}. 

A different approach using social media was presented by Choudhury et
al. \cite{DeChoudhury2010} in 2010. Geo-referenced Flickr content was used to gather
information such as the date and location of the photos being
uploaded. An OP algorithm was then used to generate the ideal number
of Point of Interests (POI).




\subsection{User Profiling for Travel Preferences}
 