\subsection{Problem Definition}

        Producing an itinerary before a trip can be a demanding task
        which requires  a substantial amount of  research. Many times
        people rely on travel books, individual travel blogs and
        online websites to form a holiday plan, but these are not
        always tailored according to the traveller’s preferences and
        opinions \cite{DeChoudhury2010}. 

        This paper focuses on creating a system which helps tourists
        automate the process of travel planning. An adequate automated
        trip planner application would consist of two parts, 
        
        \begin{enumerate}
                \item the retrieval of user preferences 
                \item the generation of a custom itinerary
        \end{enumerate}
        Numerous systems, which will be discussed in the Literature
        Review, are available and therefore building a working
        prototype is both possible and feasible
        % \cite{Sylejmani2017,Chang2016,Sylejmani2012,Sebastia2009a,
        % Tumas2009,Kurata2013, RamalhoBrilhante2014,
        % DeChoudhury2010,DUNSTALL2008a, DiBitonto2010a,Gavalas }.
        Although these systems automate the process of producing the
        itinerary, they require a lot of end-user data and preferences
        to form a personalised itinerary. Can the user preference
        gathering be automated?

        Given the amount of information a single user holds online, it
        is possible to automate and help the process of gathering
        personal preferences \cite{Buraya2017}. A deep learning model
        could be trained to classify a person's social media profile
        to determine what the user wants from a trip. This information
        alongside other parameters such as the user's budget and trip
        length could give out a very accurate personalised holiday
        plan.

\subsection{Motivation}

        The immense amount of data generated by each user online
        \cite{J.Clement2020} was the main motivation behind using this
        advantage in creating a unique system that benefits tourists
        by implementing something easy to use and does not bombard
        them with a lot of extra questions. Although planning
        itineraries can be a complex problem \cite{DUNSTALL2008a}, if
        the users allows the system to gather preferences based on
        their social media profile, preferences can be collected
        automatically based on his posts. 

\subsection{Why the Problem is non-trivial}

        User Profiling based on social media has been an essential
        part of Personalized advertising. The advertisers can target
        their customers more accurately and earn more sales per
        viewer\cite{article}. However, this paper aims in using such a
        technology to implement a different approach in automating 
        the preference gathering.

\subsection{Aims and Objectives}
        
        The aim of this project is to quickly generate a personalised
        itinerary by making use of preferences and parameters. 

        This system will aim to achieve the following Objectives:
        \begin{enumerate}
                \item Collect social media images to form a training
                and testing set which will be categoriesd by the
                activity. These can include images associated with
                events such as, nature, beach, sports, food, bars and
                clubs.
                \item Design a model that classifies the images
                correctly.
                \item Define a user profile based on the social media
                collection results and additional parameters.
                \item Gather a list of places available and form
                scores for each activity based the user's parameters.
                \item Generate quickly multiple itineraries each with
                different score levels.
        \end{enumerate}