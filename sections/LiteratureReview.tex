Several studies both on user profiling and on real-time automatic trip
itinerary generation have been carried out throughout the years. There
are many types of systems which help the travellers in their trips.
Gavalas et al. \cite{Gavalas2014} categorised these into \textbf{POI
recommenders, Tourist Service Recommenders, Collaborative content from
users and social media services, path recommenders} and
\textbf{Personlised multiple-day tour planners}. The planning of a
trip to a traveller introduces the Tourist Trip Design Problem (TTDP)
which has recieved a lot of observation and heuristic contribution
\cite{Gunawan2016,Delic2018}. Sylejmani et al.\cite{Sylejmani2017}
have defined the TTDP as part of the Orienteering Problem(OP). OP
problems contain a number of nodes each containing a score and try to
solve the path containing the maximal score constrained with
parameters such as time and budget \cite{Gunawan2016}. Gunawan et al.
\cite{Gunawan2016} state that OP is a combination of the Knapsack
problem and the Travelling Salesman Problem (TSP). There are many
solutions to this problem which will be discussed in the next section.
% Maybe talk about OP..etc

\subsection{Tourist Recommender Systems}

    In 2004, a paper by Dunstall et al. \cite{DUNSTALL2008a} was
    published using a prototype called the The Electronic Travel
    Planner (ETP). This system selects destinations by determining
    activities based on the user’s preferences. Each activity is
    stored in a relational database with information such as duration,
    availability, date and time categorised as either tours, lodging
    or transportation. The requirements for forming such an itinerary
    include the number of children and adults, the location, the date
    range, budget and user preferences in the form of \emph{mandatory,
    at least once, desired, forbidden and permitted} activities. Since
    examples given in the paper took 15-45 seconds to process the
    resulting running time was listed as an issue.

    The Recommender System (RS) was provided by Sebatsia et al.
    \cite{Sebastia2009a} and Garcia et al. \cite{Garcia2011} to
    suggest tourist locations. User preferences are collected in the
    form of age, gender, nationality and ontology. The recommender is
    based on 5 techniques, \emph{Demographic recommendation,
    collaborative recommendation, content based recommendation and
    knowledge based recommendation}. 

    A different approach using social media was presented by Choudhury
    et al. \cite{DeChoudhury2010} in 2010 and Brilhante et al.
    \cite{RamalhoBrilhante2014}. Geo-referenced Flickr
    \footnote{\url{https://www.flickr.com/}} content alongside
    Wikipedia \footnote{\url{https://www.wikipedia.org/}} information
    was used to gather information such as the date, location and
    popularity of the photos being uploaded. An OP algorithm was then
    used to generate the ideal number of Point of Interests (POI).

    A tabu Search approach was proposed by Sylejmani et al.
    \cite{Sylejmani2012} as a Multi Constrained Team Orienteering
    Problem with Time Windows (\textbf{MCTOPTW}), an advanced form of
    the OP.In this algorithm, three steps were used in order to
    generate the activity plan. A new activity is added as a node to
    the trip using \emph{Insert}, A node is exchanged with a new
    activity using \emph{Replace} and two nodes are swapped using
    \emph{Swap}. A pair of tabu lists structured frequently are used
    to avoid repeating solutions.


    Recently, a solution towards presenting an itinerary solving
    conflicts between multiple tourists with different preferences by
    creating a group recommender system (GRS)
    \cite{Castro2015,Delic2018} was proposed by  \cite{Sylejmani2017}.
    All tourists are split into groups by preference, during certain
    activities the itinerary splits up the groups to visit their
    specific POI. Before the trip one of the options is selected: 

    \begin{enumerate} 
        \item \textbf{Solo}: A trip for a single person. 
        \item \textbf{Subgroups}: The tourists are separated into smaller groups by preference and travel together.
        \item \textbf{All Together}: One itinerary for all Tourists.
        \item \textbf{Tourists Combined}: At certain times, tourists
    are separated to meet their personal preferences
    \end{enumerate}

    An unique approach towards collecting the group's preferences for
    a GRS is offered by Nguyen et al.\cite{Nguyen2018} An android
    group chat application called STSGroup was created to target
    conflicts between tourists. The idea is to collect the users’
    preferences when they are communicating with each other rather
    than individually. An example of a group of students travelling to
    South Tylrol (Italy) was given in the paper in which each person
    described its profile using certain tags such as the mood or parts
    of groups they form of. Upon the text conversation users send a
    selected POI in which other users can give it a thumbs up or down.
    Ranking lists and logistics are calculated in the background based
    on the group chat's data collection to determine the ideal
    preferences for the group.

    Iterated Local Search could be used to generate the travel
    itineraries as seen vansteenwegen et al 2009
    \cite{Vansteenwegen2009}. This approach is considered to be very
    suitable for real-time TTDP applications \cite{Gavalas2015}. This
    approach finds a solution using the the best generated outputs
    from local search and repeats the procedure until a desired score
    is reached. In 2011, CityPlanner3 \cite{Vansteenwegen2011a}
    integrated ILS with Greedy Randomised adaptive search Procedure
    (GRASP) \cite{Feo1995}. This system allowed to alter POI durations
    and choose a starting and ending points

\subsection{User Profiling for Travel Preferences}
    Recent years have shown how the average internet user has gone
    from a passive content absorber to also content producer through
    the rise in social media \cite{Ikeda}. This section describes
    methods of user profiling and information gathering.\\

    \subsubsection{User Profiling based on textual methods} Textual
    data from comments and posts can be used to gather such user
    preferences. In 2013, A system was shown by Ikeda et al.
    \cite{Ikeda} which could perform sentiment analysis based on
    100,000 Japanese user profiles and perform demographic estimation.
    Tags from social media posts can also be useful information to
    gather information about the user. Hung et al. \cite{Hung2008}
    demonstrated a technique in user profiling based on tags.  Given
    an object and a user, the similarity between the tags is
    calculated. Both of these maximum similarities are summed up by
    the correlation between the set of object tags and the set of user
    tags. \\

    \subsubsection{User Profiling based on images} Terttunen
    \cite{Terttunen2017} has shown how Instagram
    \footnote{https://www.instagram.com/} has been of a major
    influence towards tourists. The ability to share photos of the
    amazing sights and landscapes has provided with more excitement
    than looking for inspiration in a tourism brochure.

    Chen et al. \cite{Chen2013} produced a system for automatically
    retrieving tags from images and incomplete tags called
    \emph{FastTag}. The algorithm can be trained in \(O(n)\) time and
    uses two simple linear mappings. Figure \ref{fasttag} shows an
    example of an input image used alongside the incomplete input tags
    \emph{snow, lake, feet}. Given these two inputs the algorithm was
    able to produce the following tags, {mountain, snow, sky, lake, water,feet, legs,
    boat, trees}.
    \begin{figure}[H]
        \caption{The image shows an example given in the FastTag
        Paper.\cite{Chen2013}}
        \centering
        \includegraphics{fasttagSample}
        \label{fasttag}
    \end{figure}
