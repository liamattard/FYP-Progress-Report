Several studies both on user profiling and on real-time automatic trip
itinerary generation have been carried out throughout the years. 

There are many types of systems which help the travellers in their
trips. Gavalas et al. \cite{Gavalas2014} categorised these into
\textbf{POI recommenders, Tourist Service Recommenders, Collaborative
content from users and social media services, path recommenders} and
\textbf{Personlised multiple-day tour planners}. The planning of a
trip to a traveller introduces the Tourist Trip Design Problem (TTDP)
which has recieved a lot of observation and heuristic contribution
\cite{Gunawan2016,Delic2018}. Sylejmani et al.\cite{Sylejmani2017} have defined
the TTDP as part of the Orienteering Problem(OP). OP problems contain
a number of nodes each containing a score and try to solve the path
containing the maximal score constrained with parameters such as time
and budget \cite{Gunawan2016}. There are many solutions to this
problem which will be discussed in the next section.
% Maybe talk about OP..etc

\subsection{Tourist Recommender Systems}

    In 2004, a paper by Dunstall et al. \cite{DUNSTALL2008a} was published
    using a prototype called the The Electronic Travel Planner (ETP). This
    system selects destinations by determining activities based on the
    user’s preferences. Each activity is stored in a relational database
    with information such as duration, availability, date and time
    categorised as either tours, lodging or transportation. The
    requirements for forming such an itinerary include the number of
    children and adults, the location, the date range, budget and user
    preferences in the form of \emph{mandatory, at least once, desired,
    forbidden and permitted} activities. Since examples given in the paper
    took 15-45 seconds to process the resulting running time was listed as
    an issue.

    The Recommender System (RS) was provided by Sebatsia et al.
    \cite{Sebastia2009a} and Garcia et al. \cite{Garcia2011} to suggest
    tourist locations. User preferences are collected in the form of age,
    gender, nationality and ontology. The recommender is based on 4
    techniques, \emph{Demographic recommendation, collaborative
    recommendation, content based recommendation and knowledge based
    recommendation}. 

    A different approach using social media was presented by Choudhury et
    al. \cite{DeChoudhury2010} in 2010 and Brilhante et al.
    \cite{RamalhoBrilhante2014}. Geo-referenced Flickr
    \footnote{\url{https://www.flickr.com/}} content alongside Wikipedia
    \footnote{\url{https://www.wikipedia.org/}} information was used to
    gather information such as the date, location and popularity of the
    photos being uploaded. An OP algorithm was then used to generate the
    ideal number of Point of Interests (POI).

    A tabu Search approach was proposed by Sylejmani et al.
    \cite{Sylejmani2012} as a Multi Constrained Team Orienteering Problem
    with Time Windows (\textbf{MCTOPTW}), an advanced form of the OP.In
    this algorithm, three steps were used in order to generate the
    activity plan. A new activity is added as a node to the trip using
    \emph{Insert}, A node is exchanged with a new activity using
    \emph{Replace} and two nodes are swapped using \emph{Swap}. A pair of
    tabu lists structured frequently are used to avoid repeating
    solutions.


    Recently, a solution towards presenting an itinerary solving
    conflicts between multiple tourists with different preferences was
    proposed by  \cite{Sylejmani2017}. All tourists are split into
    groups by preference, during certain activities the itinerary
    splits up the groups to visit their specific POI. Before the trip
    one of the options is selected: 
    \begin{enumerate}
        \item \textbf{Solo}: A trip for a single person.
        \item \textbf{Subgroups}: The tourists are separated into
        smaller groups by preference and travel together.
        \item \textbf{All Together}: One itinerary for all Tourists.
        \item \textbf{Tourists Combined}: At certain times, tourists are separated to
    meet their personal preferences

    \end{enumerate}





\subsection{User Profiling for Travel Preferences}
 