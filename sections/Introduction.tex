\subsection{Problem Definition}

        Producing an itinerary before a trip can be a demanding task
        which requires  a substantial amount of  research. Many times
        people rely on travel books, individual travel blogs and
        online websites to form a holiday plan, but these are not
        always tailored according to the traveller’s preferences and
        opinions \cite{DeChoudhury2010}. 

        An adequate automated trip planner application would consist
        of two parts, 
        
        \begin{enumerate}
                \item the retrieval of user preferences 
                \item the generation of a custom itinerary
        \end{enumerate}
        Numerous systems are available and therefore building a
        working prototype is both possible and feasible
        \cite{Sylejmani2017,Chang2016,Sylejmani2012,Sebastia2009a,
        Tumas2009, Vansteenwegen2011,Kurata2013, RamalhoBrilhante2014,
        DeChoudhury2010,DUNSTALL2008a, DiBitonto2010a,Gavalas }.
        Although these systems automate the process of producing the
        itinerary, they require a lot of end-user data and preferences
        to form a personalised itinerary. Can the user preference
        gathering be automated?

        Given the amount of information a single user holds online, it
        is possible to automate and help the process of gathering
        personal preferences \cite{Buraya2017}. A deep learning model
        could be trained to classify a person's social media profile
        to determine what the user wants from a trip. This information
        alongside other parameters such as the user's budget and trip
        length could give out a very accurate personalised holiday
        plan.

\subsection{Motivation}
        The immense amount of data generated by each user online
        \cite{J.Clement2020} was the main motivation behind using such
        an advantage in creating a system that benefits tourists by
        implementing something easy to use which does not bombard them
        with a lot of extra questions. Although planning itineraries
        can be a complex problem \cite{DUNSTALL2008a}, if the users
        allows the system to gather preferences based on their social
        media profile, a personalised itinerary could be generated
        possibly reducing the ammount of preferences to ask the
        tourists.

\subsection{Why the Problem is non-trivial}
        User Profiling has been an essential part of Personalized
        advertising since advertisers can pick out customers more
        accurately. 


\subsection{Aims and Objectives}