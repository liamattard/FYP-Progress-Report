\subsection{Problem Definition}

        Producing an itinerary before a trip can be a demanding task
        which requires  a substantial amount of  research. Many times
        people rely on travel books, individual travel blogs and online
        websites to form a holiday plan, but these end up influencing 
        traveller’s preferences and opinions \cite{DeChoudhury2010}. 
        % Ask about this citation

        An adequate automated trip planner application would consist
        of two parts, data retrieval based on the end-user and a
        generated itinerary. Numerous systems are available and 
        therefore building a working prototype is possible and feasible 
        \cite{Sylejmani2017,Tumas2009, Vansteenwegen2011,Kurata2013,
        RamalhoBrilhante2014, DeChoudhury2010,DUNSTALL2008a, 
        DiBitonto2010a,Gavalas}. Altough these systems can automate well the
        process of producing the itinerary, do they require a lot of 
        end-user data and preferences? How much of the user information
        gathering can be automated?

        Given the amount of information a single user holds online, 
        it is possible to automate and help the process of 
        gathering personal preferences \cite{Buraya2017}. A deep 
        learning model could be trained to classify a person's social
        media profile, for instance Instagram and 

\subsection{Motivation}